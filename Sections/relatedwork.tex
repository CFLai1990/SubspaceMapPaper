\section{Related Work}

In this section, we first take a look at the general high-dimensional data visualizations to see how people usually choose the important dimensions. Then we introduce the subspace mining techniques, along with the related visualization works. At last, we discuss the advantages and disadvantage of the map based visualizations by briefly reviewing the visualizations using the map metaphors.

\subsection{Selecting Dimensions in a Visualization}

Choosing the important ones from many dimensions is never an easy task. It requires the analyst to have a rich experience with the data, which is seldom the case in data analysis. If the analysts themselves are the data collectors, they might know well about the semantics of the dimensions. But it still takes time to get familiar with the data distribution, not to mention some dimensions could have similar semantics. A lot of research has been done to help analysts deal with this difficulty. The dimension selection approaches can be divided into three genres based on their underlying strategies.

The first kind aims to solve the similarity issue by deriving a diverse subset of dimensions. In the early research of parallel coordinates~\cite{DBLP:journals/vc/Inselberg85}, Yang et al.~\cite{DBLP:conf/infovis/YangPWR03} proposed to cluster the similar dimensions and extract a representative for each cluster. The representative is either a centroid or an average of the cluster members. Turkay et al.~\cite{DBLP:journals/tvcg/TurkayLLH12} added another two ways to generate the representatives: dimension reduction and statistical modeling. Zhang et al.~\cite{DBLP:conf/apvis/ZhangMM12, DBLP:journals/tvcg/ZhangMZM15} made use of the correlation strengths and carried out more delicate rules for dimension clustering. 

The second kind aims to solve the importance issue by deriving a critical or informative subset of dimensions. It can be traced back to the Rank-by-Feature Framework~\cite{SeoS05}, which ranks histograms of all dimensions based on user-selected metrics. The idea was later applied to choose dimension pairs in a scatterplot matrix (SPLOM)~\cite{carr1987scatterplot} and order axes in parallel coordinates. For scatterplots, there are many metrics~\cite{DBLP:journals/cgf/JohanssonC08, DBLP:journals/cgf/SipsNLH09, DBLP:conf/ieeevast/TatuAESTMK09} to measure the saliency of data patterns, including the well-known Scagnostics~\cite{DBLP:conf/infovis/WilkinsonAG05, DBLP:journals/tvcg/WilkinsonAG06}. They are used to rank the large amount of plots in a SPLOM~\cite{DBLP:journals/ivs/Guo03, DBLP:conf/apvis/NhonW14}, in order to improve the efficiency of exploration. For parallel coordinates, metrics are also proposed to detect the inter-axis patterns~\cite{DBLP:journals/cgf/JohanssonC08, DBLP:journals/tvcg/DasguptaK10} and rank different ordering schemes~\cite{DBLP:conf/infovis/PengWR04, DBLP:journals/tvcg/JohanssonJ09}. For a more comprehensive overview, please refer to~\cite{DBLP:journals/tvcg/Bertini11}.

As opposed to the automatic methods, the third kind facilitates interactive dimension selection without assuming any predefined metrics of interestingness. Voyager~\cite{DBLP:journals/tvcg/WongsuphasawatM16} and its advanced version~\cite{DBLP:conf/chi/WongsuphasawatQ17} allow free selection from a complete list and recommend dimensions that may have been overlooked. Sarvghad et al.~\cite{DBLP:journals/tvcg/SarvghadTM17} achieve the same goal by simply showing the explored dimension coverage. Turkay et al.~\cite{DBLP:journals/tvcg/TurkayFH11} proposed to brush dimensions in a projection to support dual-space analysis. It was later extended into a more scalable framework by Yuan et al.~\cite{DBLP:journals/tvcg/YuanRWG13}, which organizes the exploration in a hierarchical way.

\subsection{Subspace Mining and Visualization}

The dimension selection approaches work well to identify 1D or 2D subspaces. But when it comes to higher-dimensional situations, subspace mining techniques are often indispensable. They are designed to find subspaces with interesting data patterns, which in most cases refer to hidden data clusters. Hence, such techniques are also called Subspace Clustering~\cite{DBLP:conf/sigmod/AgrawalGGR98}. In the last few decades, people have developed various kinds of methods in this domain~\cite{DBLP:journals/spm/Vidal11}. We will not elaborate on the technical details. Instead, we introduce the visualization works supported by subspace clustering, since they are more relevant to our S-Map approach.

Like other heuristic algorithms, subspace clustering generates multiple results and easily causes redundancy. Tatu et al.~\cite{DBLP:conf/ieeevast/TatuMFBSSK12} developed a visual analytic system to help users organize the redundant subspace candidates. It shows the relationships of all candidates in a projection by comparing their data topology. TripAdvisor\({}^{\mbox{N-D}}\)~\cite{DBLP:journals/tvcg/NamM13} shows a similar projection that measures the dimensional differences. Jäckle et al.~\cite{jackle2017pattern} adopts a 1D layout to help users trace data changes across different subspaces. Watanabe et al.~\cite{DBLP:conf/apvis/WatanabeWNTF15} did not depend on subspace clustering, but invented a bi-clustering algorithm to divide the data into multiple complementary subparts.

Among the above methods,~\cite{DBLP:conf/ieeevast/TatuMFBSSK12} and~\cite{DBLP:journals/tvcg/NamM13} both have a main view similar to our S-Map. But instead of projections, S-Map uses a more compact and scalable layout to avoid visual occlusion.~\cite{jackle2017pattern} shares with us the same goal to reveal the impacts of dimensions on data structures. However, it only studies projected data changes that suffer from information loss and are susceptible to geometric transformations, while we focus on structural differences in the original subspaces. Besides, 1D alignments are much less competent in visualizing relationships when compared to 2D layouts. Most of all, these methods only visualize a small fraction of subspaces found by the mining algorithms, while S-Map provides an overview of the whole parameter space.

\subsection{Map-Based Visualizations}

A map (or more formally, cartography~\cite{kraak2013cartography}) is a very ancient, popular and important technique to visualize spatial relationships in the world. Most people are familiar with maps and often rely on maps to know a strange environment. It makes them highly efficient in conveying unfamiliar information. That's why maps are not only widely applied in spatial data visualization~\cite{DBLP:journals/ivs/AndrienkoADFW08}, but often used as a carrier of non-spatial information.

The map-like visualizations can be categorized into two types: irregular maps and mosaic maps, depending on whether the map units are uniform polygons. GMap~\cite{DBLP:conf/apvis/GansnerHK10} is a typical irregular map formed by voronoi cells. It was designed to show communities in a network, and later applied to visualize dynamic graphs~\cite{DBLP:journals/tvcg/MashimaKH12} and social media data~\cite{DBLP:journals/jgaa/GansnerHN13}. Chen et al.~\cite{chen2017map} introduced more geographic concepts like cities and rivers to depict the evolution of social media events. For mosaic maps, Cao et al.~\cite{DBLP:journals/tvcg/CaoLG16} proposed to visualize multi-label data with ternary plots in uniform triangular grids. Closer to our approach are the works that use hexagon tiling~\cite{DBLP:conf/vinci/YangB15} to promote various visual analytic tasks, such as cluster comparison~\cite{DBLP:journals/cgf/RieckL16}, social media analysis~\cite{DBLP:conf/ieeevast/ChenCWLYCW16} and so on. Also built upon hexagon tiles, S-Map supports more abundant map-based semantics and more scalable explorations.