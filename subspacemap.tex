% $Id: template.tex 11 2007-04-03 22:25:53Z jpeltier $

%\documentclass{vgtc}                          % final (conference style)
\documentclass[review]{vgtc}                 % review
%\documentclass[widereview]{vgtc}             % wide-spaced review
%\documentclass[preprint]{vgtc}               % preprint
%\documentclass[electronic]{vgtc}             % electronic version
% \let\ifpdf\relax

%% Uncomment one of the lines above depending on where your paper is
%% in the conference process. ``review'' and ``widereview'' are for review
%% submission, ``preprint'' is for pre-publication, and the final version
%% doesn't use a specific qualifier. Further, ``electronic'' includes
%% hyperreferences for more convenient online viewing.

%% Please use one of the ``review'' options in combination with the
%% assigned online id (see below) ONLY if your paper uses a double blind
%% review process. Some conferences, like IEEE Vis and InfoVis, have NOT
%% in the past.

%% Please note that the use of figures other than the optional teaser is not permitted on the first page
%% of the journal version.  Figures should begin on the second page and be
%% in CMYK or Grey scale format, otherwise, colour shifting may occur
%% during the printing process.  Papers submitted with figures other than the optional teaser on the
%% first page will be refused.

%% These three lines bring in essential packages: ``mathptmx'' for Type 1
%% typefaces, ``graphicx'' for inclusion of EPS figures. and ``times''
%% for proper handling of the times font family.

\usepackage{mathptmx}
\usepackage{graphicx}
\usepackage{times}
\usepackage{enumerate}
\usepackage{color}
\usepackage{bm}
\usepackage{amsmath}
\usepackage{subfigure}

%% We encourage the use of mathptmx for consistent usage of times font
%% throughout the proceedings. However, if you encounter conflicts
%% with other math-related packages, you may want to disable it.

%% If you are submitting a paper to a conference for review with a double
%% blind reviewing process, please replace the value ``0'' below with your
%% OnlineID. Otherwise, you may safely leave it at ``0''.
\onlineid{0}

%% We encourage the use of mathptmx for consistent usage of times font
%% throughout the proceedings. However, if you encounter conflicts
%% with other math-related packages, you may want to disable it.

%% This turns references into clickable hyperlinks.
%% If you are submitting a paper to a conference for review with a double
%% blind reviewing process, please replace the value ``0'' below with your
%% OnlineID. Otherwise, you may safely leave it at ``0''.
\onlineid{0}
\newcommand{\note}[1]{\iffalse #1 \fi}
\newcommand{\col}[1]{{\color{black}{#1}}}

\tolerance=1
\emergencystretch=\maxdimen
\hyphenpenalty=10000
\hbadness=10000
%% declare the category of your paper, only shown in review mode
\vgtccategory{Research}

%% allow for this line if you want the electronic option to work properly
\vgtcinsertpkg

%% In preprint mode you may define your own headline.
%\preprinttext{To appear in an IEEE VGTC sponsored conference.}

%% Paper title.

\title{S-Map: Subspace Map for High Dimensional Data Visualization}

%% This is how authors are specified in the journal style

%% indicate IEEE Member or Student Member in form indicated below
\author{Chufan Lai\textsuperscript{1}\thanks{e-mail: chufan.lai@pku.edu.cn} %
\and Xiaoru Yuan\textsuperscript{1, 2}\thanks{e-mail: xiaoru.yuan@pku.edu.cn}}


\affiliation{\scriptsize
 1) Key Laboratory of Machine Perception (Ministry of Education), and School of EECS, Peking University\\
 2) Beijing Engineering Technology Research Center of Virtual Simulation and Visualization, Peking University}

%other entries to be set up for journal
%% \shortauthortitle{Biv \MakeLowercase{\textit{et al.}}: Global Illumination for Fun and Profit}
%\shortauthortitle{Firstauthor \MakeLowercase{\textit{et al.}}: Paper Title}

%% Abstract section.
\abstract{
\note{
When faced with a high-dimensional dataset, analysts often focus on a subset of the dimensions, which is known as a subspace. However, the analysis of subspace data is extremely complicated. On the one hand, the subspaces grow exponentially as the dimensionality increases. It's impossible to check on each one of them. On the other hand, data distributions could be highly different across various subspaces. it's a tough task to compare them and understand the impacts of dimension selection to the final data structures.
In this paper, we propose Subspace Map, a novel visualization using the map metaphor to summarize the features of subspaces and guide users in the exploration. To be specific, we sample the subspaces and regard them as 'cities'. Multiple cities are clustered into higher-level regions based on their data similarities. For each region/cluster, we extract its featured dimensions and stable structures to reflect the impacts of dimension combinations to the data distribution. On different clustering levels, we pick out the representative subspaces as 'capitals' and build up a traffic system between them to guide users in the tour. Case studies with real-world datasets are carried out to prove the effectiveness of our Subspace Map system.
}
Analysis of subspaces of a complex high-dimensional dataset is extremely challenging, as the number of subspaces grows exponentially as the dimensionality increases and the data distributions could be highly different across various subspaces. 
In this paper, we propose S-Map, e.g. Subspace Map, a novel visualization approach with the map metaphor to summarize the features of subspaces and guide users in efficient exploration. In S-Map, subspaces are clustered into continents or islands depends on their population and similarity. Representative subspaces are identified as cities on the map. Altitude of the lands reflects the density of the clusters. For each map region, we extract its featured dimensions and stable structures to reflect the impacts of dimension combinations to the data distribution. On different clustering levels, we pick out the representative subspaces as 'capitals' and build up a traffic system between them to guide users in the tour. Case studies with real-world datasets are presented to demonstrate the effectiveness of the proposed system.
} % end of abstract

%% Keywords that describe your work. Will show as 'Index Terms' in journal
%% please capitalize first letter and insert punctuation after last keyword
\keywords{High-Dimensional Data Analysis, Subspace Exploration, Map Metaphor}

%% ACM Computing Classification System (CCS).
%% See <http://www.acm.org/class/1998/> for details.
%% The ``\CCScat'' command takes four arguments.
% \note{
% \CCScatlist{ % not used in journal version
%  \CCScat{K.6.1}{Management of Computing and Information Systems}%
% {Project and People Management}{Life Cycle};
%  \CCScat{K.7.m}{The Computing Profession}{Miscellaneous}{Ethics}
% }
% }

%% Uncomment below to include a teaser figure.

%% Uncomment below to disable the manuscript note
%\renewcommand{\manuscriptnotetxt}{}

%% Copyright space is enabled by default as required by guidelines.
%% It is disabled by the 'review' option or via the following command:
% \nocopyrightspace

%%%%%%%%%%%%%%%%%%%%%%%%%%%%%%%%%%%%%%%%%%%%%%%%%%%%%%%%%%%%%%%%
%%%%%%%%%%%%%%%%%%%%%% START OF THE PAPER %%%%%%%%%%%%%%%%%%%%%%
%%%%%%%%%%%%%%%%%%%%%%%%%%%%%%%%%%%%%%%%%%%%%%%%%%%%%%%%%%%%%%%%%

\begin{document}

%% The ``\maketitle'' command must be the first command after the
%% ``\begin{document}'' command. It prepares and prints the title block.

%% the only exception to this rule is the \firstsection command

%% \section{Introduction} %for journal use above \firstsection{..} instead
\firstsection{Introduction}
\maketitle

%Paragraph 1:   what is a subspace and why we are interested in its analysis
High-dimensional data is featured by the existence of multiple attributes to describe the same data object. However, some of the attributes may not be as informative as the others. What's worse, these redundant attributes may bury deep some important data structures and relationships. For example, when analyzing the connection between education level and income, the physical attributes of a person (such as weight and height) are not very helpful. The more time we spend in studying the physical features, the less likely it is to reveal the information of interest. Therefore, experienced analysts usually pick out a small dimension subset that is most related to the current task before going deep into the details. The data space formed by these partial dimensions is called a \textbf{subspace}.

%Paragraph 2:   clarify the notion of "subspace"
In fact, subspaces can be divided into two types: axis-aligned and non-axis-aligned. Axes of the former are parallel to the original dimensions, while axes of the latter are weighted sums of the original ones. Linear dimension-reduced projections~\cite{fodor2002survey} are a typical class of non-axis-aligned subspaces. Some researchers also propose the notion of "data subspace"~\cite{DBLP:journals/tvcg/YuanRWG13}, which is a subset of data items. But in this paper, we only focus on the axis-aligned dimensional subspaces and refer to them as "subspaces" for short.

%Paragraph 3:   what is so difficult about subspace analysis
Subspace analysis could be extremely complicated due to three reasons. 
\begin{enumerate} [1).]
\item First, the exploration space is simply too huge. Choosing a subspace in an n-dimensional data is a combinatorial problem. There are altogether $2^{n}-1$ possible subspaces in different dimensionalities. It is impossible for automatic search to be successfully applied, not to mention manual analysis.
\item Second, it's hard to compare the data in different dimension settings. Data structures could vary greatly across subspaces. Comparing them requires a common, accurate and reliable measurement that is irrelevant to the specific dimensions.
\item Last but not least, it's difficult to understand and anticipate the impacts of dimension combinations to the data distributions. Analysts often choose dimensions merely based on their semantics. Including/Excluding a single dimension may seem like a subtle change, but it could cause the data structures to change dramatically. Some data patterns only appear when certain dimensions are combined. Without knowing about these complex interplays, it's difficult to find out a proper subspace with the desired data features.
\end{enumerate}

%Paragraph 4:   what have been done in the past to facilitate subspace analysis and what are the major defects


%Paragraph 5:   what inspire this work and what are our goals
%Paragraph 6:   what have been done in this work and what are the contributions
%Paragraph 7:   the overall structure of this paper
\section{Related Work}

In this section, we first take a look at the general high-dimensional data visualizations to see how people usually choose the important dimensions. Then we introduce the subspace mining techniques, along with the related visualization works. At last, we discuss the advantages and disadvantage of the map based visualizations by briefly reviewing the visualizations using the map metaphors.

\subsection{Selecting Dimensions in a Visualization}

Choosing the important ones from many dimensions is never an easy task. It requires the analyst to have a rich experience with the data, which is seldom the case in data analysis. If the analysts themselves are the data collectors, they might know well about the semantics of the dimensions. But it still takes time to get familiar with the data distribution, not to mention some dimensions could have similar semantics. A lot of research has been done to help analysts deal with this difficulty. The dimension selection approaches can be divided into three genres based on their underlying strategies.

The first kind aims to solve the similarity issue by deriving a diverse subset of dimensions. In the early research of parallel coordinates~\cite{DBLP:journals/vc/Inselberg85}, Yang et al.~\cite{DBLP:conf/infovis/YangPWR03} proposed to cluster the similar dimensions and extract a representative for each cluster. The representative is either a centroid or an average of the cluster members. Turkay et al.~\cite{DBLP:journals/tvcg/TurkayLLH12} added another two ways to generate the representatives: dimension reduction and statistical modeling. Zhang et al.~\cite{DBLP:conf/apvis/ZhangMM12, DBLP:journals/tvcg/ZhangMZM15} made use of the correlation strengths and carried out more delicate rules for dimension clustering. 

The second kind aims to solve the importance issue by deriving a critical or informative subset of dimensions. It can be traced back to the Rank-by-Feature Framework~\cite{SeoS05}, which ranks histograms of all dimensions based on user-selected metrics. The idea was later applied to choose dimension pairs in a scatterplot matrix (SPLOM)~\cite{carr1987scatterplot} and order axes in parallel coordinates. For scatterplots, there are many metrics~\cite{DBLP:journals/cgf/JohanssonC08, DBLP:journals/cgf/SipsNLH09, DBLP:conf/ieeevast/TatuAESTMK09} to measure the saliency of data patterns, including the well-known Scagnostics~\cite{DBLP:conf/infovis/WilkinsonAG05, DBLP:journals/tvcg/WilkinsonAG06}. They are used to rank the large amount of plots in a SPLOM~\cite{DBLP:journals/ivs/Guo03, DBLP:conf/apvis/NhonW14}, in order to improve the efficiency of exploration. For parallel coordinates, metrics are also proposed to detect the inter-axis patterns~\cite{DBLP:journals/cgf/JohanssonC08, DBLP:journals/tvcg/DasguptaK10} and rank different ordering schemes~\cite{DBLP:conf/infovis/PengWR04, DBLP:journals/tvcg/JohanssonJ09}. For a more comprehensive overview, please refer to~\cite{DBLP:journals/tvcg/Bertini11}.

As opposed to the automatic methods, the third kind facilitates interactive dimension selection without assuming any predefined metrics of interestingness. Voyager~\cite{DBLP:journals/tvcg/WongsuphasawatM16} and its advanced version~\cite{DBLP:conf/chi/WongsuphasawatQ17} allow free selection from a complete list and recommend dimensions that may have been overlooked. Sarvghad et al.~\cite{DBLP:journals/tvcg/SarvghadTM17} achieve the same goal by simply showing the explored dimension coverage. Turkay et al.~\cite{DBLP:journals/tvcg/TurkayFH11} proposed to brush dimensions in a projection to support dual-space analysis. It was later extended into a more scalable framework by Yuan et al.~\cite{DBLP:journals/tvcg/YuanRWG13}, which organizes the exploration in a hierarchical way.

\subsection{Subspace Mining and Visualization}

The dimension selection approaches work well to identify 1D or 2D subspaces. But when it comes to higher-dimensional situations, subspace mining techniques are often indispensable. They are designed to find subspaces with interesting data patterns, which in most cases refer to hidden data clusters. Hence, such techniques are also called Subspace Clustering~\cite{DBLP:conf/sigmod/AgrawalGGR98}. In the last few decades, people have developed various kinds of methods in this domain~\cite{DBLP:journals/spm/Vidal11}. We will not elaborate on the technical details. Instead, we introduce the visualization works supported by subspace clustering, since they are more relevant to our S-Map approach.

Like other heuristic algorithms, subspace clustering generates multiple results and easily causes redundancy. Tatu et al.~\cite{DBLP:conf/ieeevast/TatuMFBSSK12} developed a visual analytic system to help users organize the redundant subspace candidates. It shows the relationships of all candidates in a projection by comparing their data topology. TripAdvisor\({}^{\mbox{N-D}}\)~\cite{DBLP:journals/tvcg/NamM13} shows a similar projection that measures the dimensional differences. Jäckle et al.~\cite{jackle2017pattern} adopts a 1D layout to help users trace data changes across different subspaces. Watanabe et al.~\cite{DBLP:conf/apvis/WatanabeWNTF15} did not depend on subspace clustering, but invented a bi-clustering algorithm to divide the data into multiple complementary subparts.

Among the above methods,~\cite{DBLP:conf/ieeevast/TatuMFBSSK12} and~\cite{DBLP:journals/tvcg/NamM13} both have a main view similar to our S-Map. But instead of projections, S-Map uses a more compact and scalable layout to avoid visual occlusion.~\cite{jackle2017pattern} shares with us the same goal to reveal the impacts of dimensions on data structures. However, it only studies projected data changes that suffer from information loss and are susceptible to geometric transformations, while we focus on structural differences in the original subspaces. Besides, 1D alignments are much less competent in visualizing relationships when compared to 2D layouts. Most of all, these methods only visualize a small fraction of subspaces found by the mining algorithms, while S-Map provides an overview of the whole parameter space.

\subsection{Map-Based Visualizations}

A map (or more formally, cartography~\cite{kraak2013cartography}) is a very ancient, popular and important technique to visualize spatial relationships in the world. Most people are familiar with maps and often rely on maps to know a strange environment. It makes them highly efficient in conveying unfamiliar information. That's why maps are not only widely applied in spatial data visualization~\cite{DBLP:journals/ivs/AndrienkoADFW08}, but often used as a carrier of non-spatial information.

The map-like visualizations can be categorized into two types: irregular maps and mosaic maps, depending on whether the map units are uniform polygons. GMap~\cite{DBLP:conf/apvis/GansnerHK10} is a typical irregular map formed by voronoi cells. It was designed to show communities in a network, and later applied to visualize dynamic graphs~\cite{DBLP:journals/tvcg/MashimaKH12} and social media data~\cite{DBLP:journals/jgaa/GansnerHN13}. Chen et al.~\cite{chen2017map} introduced more geographic concepts like cities and rivers to depict the evolution of social media events. For mosaic maps, Cao et al.~\cite{DBLP:journals/tvcg/CaoLG16} proposed to visualize multi-label data with ternary plots in uniform triangular grids. Closer to our approach are the works that use hexagon tiling~\cite{DBLP:conf/vinci/YangB15} to promote various visual analytic tasks, such as cluster comparison~\cite{DBLP:journals/cgf/RieckL16}, social media analysis~\cite{DBLP:conf/ieeevast/ChenCWLYCW16} and so on. Also built upon hexagon tiles, S-Map supports more abundant map-based semantics and more scalable explorations.
\section{Overview}
\section{Method}
\section{Case Study}
\section{Discussion}
\section{Conclusion}

%% if specified like this the section will be committed in review mode
\acknowledgments{
}

\bibliographystyle{abbrv}
%%use following if all content of bibtex file should be shown
%\nocite{*}
\bibliography{subspacemap}
\end{document}
